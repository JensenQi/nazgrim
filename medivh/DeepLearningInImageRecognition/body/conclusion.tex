\BiAppendixChapter{结\quad 论}{conclusion}
本文的主要目的在于介绍深度学习在图像识别中的应用,文中有两条主线贯穿全文。我们先从控制论与机器学习的关系说起,随后引入了第一个刻画数据分布的模型即受限玻尔兹曼机,为了介绍受限玻尔兹曼机的训练方法,我们讨论了马尔可夫链蒙特卡罗方法。在受限玻尔兹曼机的基础上,我们讨论了如何利用它们堆叠得到深度置信网络,并介绍了反向传播算法。至此我们完成了深度学习的第一条主线即深度置信网络的讨论。随后我们展开了第二条主线即卷积神经网络的讨论,并在其后介绍了神经网络的设计技巧与GPU高性能计算。

最后,我们通过三个实验测试了深度学习在图像识别中的识别效果。在MNIST数据集中,使用深度置信网络实现了98.7\%的识别正确率,使用卷积神经网络实现了98.9\%的识别正确率。在CIFAR-10数据集中,我们使用卷积神经网络实现了62\%的识别正确率,尽管这个结果与当前世界顶尖的结果91\%的正确率相差较远,但通过与Caffe训练的卷积网络做对比,我们乐观地认为我们的网络仍有很大的收敛空间。

深度学习作为一种特征学习,较传统模式识别方法有本质化的改变,尤其是在图像识别领域。这套方法应当值得研究人员关注。