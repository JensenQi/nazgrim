% !Mode:: "TeX:UTF-8" 

\BiAppendixChapter{致\quad 谢}{Acknowledgements}

首先我需要感谢我的导师杨旭东先生。在这个课题开始之前我和杨老师对深度学习都没有过多的接触,整个课题研究过程中我们多次进行讨论,共同学习。杨老师对我十分信任,并不过多地干预我的工作内容,让我随兴趣展开研究,只在关键地方为我指导,并且深信我的实验能成功。感谢杨老师对我的信任与支持,让我有足够多的时间与信心来深入研究这个课题,特此致敬。

其次我需要感谢计算机学院硬件实验室的陈慧鹏先生。陈老师是我大学四年的思想启蒙教师,他教会了我如何思考以及人生的意义,他有趣的《计算机组成原理》课程最终使我走上计算机的道路,如果没有遇到陈老师,或许大学四年我都找不到自己兴趣所在。陈老师为我提供了舒适的实验室条件与计算资源,在这个实验室中我完成了论文的大部分研究工作,再次感谢陈老师的支持与帮助。

特别要感谢计算机学院模式识别研究中心的刘家锋先生。我曾旁听过刘老师的《模式识别》课程,刘老师课堂上严谨的授课方式,信手拈来的公式推导使我折服。承蒙刘老师不嫌弃我学识疏浅,在课后仍为我解惑释疑,在百忙之中抽出时间帮我审阅论文,我的感激之情难以言表。

我还需要感谢计算机学院自然计算实验室的刘扬先生,刘老师的《机器学习》课程深入浅出,生动地将枯燥复杂的数学语言形象化,在年初我实验遇到瓶颈时给予我鼓励与建设性意见,并带我领略了当前机器学习的研究进展与热点,十分感谢他的无私帮助。

最后我还需要感谢在实验室遇到的几位不知名的老师,其中一位老师为我解答了机器学习与模式识别两者间的差异性,这个问题长期困扰着我。另一位与我一同讨论的老师,我俩的讨论使我对深度学习中的一些模棱两可的概念清晰化。还有多位给予我鼓励的老师,遗憾的是我并不知道他们的姓名,但我仍需要对他们表示崇高的敬意。

我的文字表达能力不强,大家从我的论文也可以感受到,文字不足以表达我的感激之情,没有以上多位老师以及我身边的同学给予我的支持,这篇论文将无法完成。最后的最后,让我再一次将我的敬意献给以上的多位老师以及同学。
