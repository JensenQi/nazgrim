% !Mode:: "TeX:UTF-8" 

\title{深度学习在图像识别中的应用}
\subject{自动化}
\stuno{1110410427}
\author{戚锦秀}
\supervisor{杨旭东}
\date{2015年6月29日}



\clearpage
\setcounter{page}{1}

\iffalse
\BiAppendixChapter{摘~~~~要}{}  %使用winedt编辑时文档结构图(toc)中为了显示摘要,故增加此句;
\fi
\cabstract{
深度学习是近年来人工智能研究领域的一个热点,这个机器学习的分支以全连接的深度置信网络和局部连接的卷积神经网络为代表,融合概率图模型、马尔可夫链蒙特卡罗方法等多个技术,其基于大规模数据而训练出的模型在声音识别、图像识别等众多领域中获得了突破性成果。本文主要研究了深度置信网络及卷积神经网络的工作原理,并在MNIST与CIFAR数据集上进行模型的训练,利用GPU与CPU的异构计算在MNIST数据集上取得了98.9\%的识别正确率,在CIFAR-10数据集上取得了62\%的识别正确率。在本文中,我们还讨论了包括降维在内的多个神经网络设计技巧以及实验中发现的一些现象。
}

\ckeywords{深度学习;受限玻尔兹曼机;深度置信网络;卷积神经网络;MCMC;GPU计算}

\eabstract{
Deep learning as a branch of machine learning which is represented by a full  connected neural network  named  Deep Belief Networks as well as a local connected network named Convolutional Neural Networks has drawn lots of attention  in the field of artificial intelligence. Multiple technologies like probabilistic graphical models and Markov Chain Monte Carlo methods have integrated into deep learning nowadays and they help deep learning make a great breakthrough in many AI tasks such as speech  and  image recognition by  training models based on large-scale data. This paper focus on the principles of Deep Belief Networks and Convolution Neural Network s, and we trained some models on the MNIST and CIFAR-10 dataset using DBNs or CNNs. As a result, we achieved 98.9\% recognition accuracy on MNIST dataset and acquires 62\% recognition accuracy on CIFAR-10 dataset with the help of GPU\& CPU-based heterogeneous computing. A number of phenomena found in our experiments and design skills of the neural networks, including dimensionality reduction, will also be discussed in this paper.
}

\ekeywords{deep learning, restricted boltzmann machines, deep belief networks, convolutional neural networks, Markov chain Monte Carlo, GPU computation}

\makecover
\clearpage 